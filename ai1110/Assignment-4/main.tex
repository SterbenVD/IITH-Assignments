\let\negmedspace\undefined
\let\negthickspace\undefined
%\RequirePackage{amsmath}
\documentclass[journal,12pt,twocolumn]{IEEEtran}
 \usepackage[utf8]{inputenc}
 \usepackage{graphicx}
 \usepackage{amsmath}
 \usepackage{mathrsfs}
\usepackage{txfonts}
\usepackage{stfloats}
\usepackage{bm}
\usepackage{cite}
\usepackage{cases}
\usepackage{subfig}
 \usepackage{amsfonts}
 \usepackage{amssymb}
 \usepackage{enumitem}
\usepackage{mathtools}
\usepackage{tikz}
\usepackage{circuitikz}
\usepackage{verbatim}
\usepackage[breaklinks=false,hidelinks]{hyperref}
\usepackage{listings}
\usepackage{calc}
\usepackage{float}
\usepackage{longtable}
\usepackage{multirow}
\usepackage{multicol}
\usepackage{color}
\usepackage{array}
\usepackage{hhline}
\usepackage{ifthen}
\usepackage{chngcntr}

\newcommand{\BEQA}{\begin{eqnarray}}
\newcommand{\EEQA}{\end{eqnarray}}
\newcommand{\define}{\stackrel{\triangle}{=}}
\bibliographystyle{IEEEtran}
%\bibliographystyle{ieeetr}
\def\inputGnumericTable{}

\let\vec\mathbf

\providecommand{\pr}[1]{\ensuremath{\Pr\left(#1\right)}}
\providecommand{\sbrak}[1]{\ensuremath{{}\left[#1\right]}}
\providecommand{\lsbrak}[1]{\ensuremath{{}\left[#1\right.}}
\providecommand{\rsbrak}[1]{\ensuremath{{}\left.#1\right]}}
\providecommand{\brak}[1]{\ensuremath{\left(#1\right)}}
\providecommand{\lbrak}[1]{\ensuremath{\left(#1\right.}}
\providecommand{\rbrak}[1]{\ensuremath{\left.#1\right)}}
\providecommand{\cbrak}[1]{\ensuremath{\left\{#1\right\}}}
\providecommand{\lcbrak}[1]{\ensuremath{\left\{#1\right.}}
\providecommand{\rcbrak}[1]{\ensuremath{\left.#1\right\}}}
\providecommand{\abs}[1]{\left\vert#1\right\vert}
\providecommand{\res}[1]{\Res\displaylimits_{#1}}
\newcommand{\myvec}[1]{\ensuremath{\begin{pmatrix}#1\end{pmatrix}}}
\newcommand{\mydet}[1]{\ensuremath{\begin{vmatrix}#1\end{vmatrix}}}

\newcommand{\question}{\noindent \textbf{Question: }}
\newcommand{\solution}{\noindent \textbf{Solution: }}


\title{Assignment 4}
\author{Vishal Vijay Devadiga (CS21BTECH11061)}
\date{}
\begin{document}
% make the title area
\maketitle
\question
A box contains 5 red marbles, 8 white marbles and 4 green marbles. One marble is taken
out of the box at random. What is the probability that the marble taken out will be:
\begin{enumerate}[label=(\roman{enumi})]
	\item Red?
	\item White?
	\item Not Green?
\end{enumerate}
\solution
\begin{table}[H]
	\input{table/table.tex}
	\caption{Distribution of Ball wrt Colour}
	\label{tab1}
\end{table}
Let's denote the outcome of the experiment by a random variable $X$ such that:
\begin{table}[H]
	\input{table/event.tex}
	\caption{Description of Events}
	\label{tab2}
\end{table}
\begin{enumerate}[label=(\roman{enumi})]
    \item $X = 0$ denotes the marble is red.
    \begin{align}
        \pr{X = 0} & = \frac{5}{17} 
	    \\
	    & = \fbox{0.294}
    \end{align}
    \item $X = 1$ denotes the marble is white.
    \begin{align}
        \pr{X = 1} & = \frac{8}{17} 
	    \\
	    & = \fbox{0.471}
    \end{align}
    \item $X \not = 2$ denotes the marble is not green, that is,
	$X \in \cbrak{0,1}$. Thus, the marble is either white or red.
    \begin{align}
        \pr{X \not = 2} & = \pr{X \in \cbrak{0,1}}
		\\
		& = \pr{X = 0} + \pr{X = 1}
		\\
		& =\frac{5+8}{17}
	    \\
	    & = \fbox{0.765}
    \end{align}
\end{enumerate}
Output of the program used to verify whether the solution is correct:
\begin{figure}[H]
	\centering
	\includegraphics[width = \columnwidth]{./figure/Output.png}
	\caption{Output of the Program}
	\label{fig1}
\end{figure}
\end{document}