\documentclass{beamer}
\usetheme{CambridgeUS}
\usepackage[utf8]{inputenc}
\usepackage{graphicx}
\usepackage{amsmath}
\usepackage{mathrsfs}
\usepackage{stfloats}
\usepackage{bm}
\usepackage{cite}
\usepackage{cases}
\usepackage{subfig}
\usepackage{amsfonts}
\usepackage{amssymb}
\usepackage{mathtools}
\usepackage{tikz}
\usepackage{circuitikz}
\usepackage{verbatim}
\usepackage{calc}
\usepackage{float}
\usepackage{longtable}
\usepackage{multirow}
\usepackage{multicol}
\usepackage{color}
\usepackage{array}
\usepackage{hhline}
\usepackage{ifthen}
\usepackage{chngcntr}
\usepackage{hyperref}

\newcommand{\BEQA}{\begin{eqnarray}}
\newcommand{\EEQA}{\end{eqnarray}}
\newcommand{\define}{\stackrel{\triangle}{=}}
\def\inputGnumericTable{}

\let\vec\mathbf

\providecommand{\pr}[1]{\ensuremath{\Pr\left(#1\right)}}
\providecommand{\sbrak}[1]{\ensuremath{{}\left[#1\right]}}
\providecommand{\lsbrak}[1]{\ensuremath{{}\left[#1\right.}}
\providecommand{\rsbrak}[1]{\ensuremath{{}\left.#1\right]}}
\providecommand{\brak}[1]{\ensuremath{\left(#1\right)}}
\providecommand{\lbrak}[1]{\ensuremath{\left(#1\right.}}
\providecommand{\rbrak}[1]{\ensuremath{\left.#1\right)}}
\providecommand{\cbrak}[1]{\ensuremath{\left\{#1\right\}}}
\providecommand{\lcbrak}[1]{\ensuremath{\left\{#1\right.}}
\providecommand{\rcbrak}[1]{\ensuremath{\left.#1\right\}}}
\providecommand{\abs}[1]{\left\vert#1\right\vert}
\providecommand{\res}[1]{\Res\displaylimits_{#1}}
\newcommand{\myvec}[1]{\ensuremath{\begin{pmatrix}#1\end{pmatrix}}}
\newcommand{\mydet}[1]{\ensuremath{\begin{vmatrix}#1\end{vmatrix}}}

\title{AI1110: Probability and Random Variables}
\subtitle{Assignment 6}
\author{Vishal Vijay Devadiga (CS21BTECH11061)}
\date{\today}
\begin{document}

% Title page frame
\begin{frame}
    \titlepage
\end{frame}

% Remove logo from the next slides
\logo{}

% Outline frame
\begin{frame}{Outline}
    \tableofcontents
\end{frame}

% Beginning of Actual Doc
\section{Question}
\begin{frame}{Question}
    Prove that if $E$ and $F$ are independent events, then so are the events $E$ and $F'$.
\end{frame}

\section{Solution}
\subsection{Data Given from Question}
\begin{frame}{Data Given from Question}
    \begin{block}{Theory}
        Probability of Intersection of 2 or more independent events is the product of probability of the events happening individually.
    \end{block}
    Given, $E$ and $F$ are independent events. Thus,
    \begin{align}
        \label{eq:1}
        \pr{E F} = \pr{E} \times \pr{F}
    \end{align}
\end{frame}

\subsection{Finding $EF'$}
\begin{frame}{Finding $EF'$}
    $F'$ and $F$ are mutually exclusive events. $E$ can be expressed such as:
    \begin{align}
        E = EF + EF'
    \end{align}
    $EF$ and $EF'$ are also mutually exclusive events. Therefore,
    \begin{align}
        \pr{E}            & = \pr{EF} + \pr{EF'}
        \\
        \label{eq:2}
        \implies \pr{EF'} & = \pr{E} - \pr{EF}
    \end{align}
\end{frame}

\subsection{Result}
\begin{frame}{Result}
    Using \eqref{eq:1} and \eqref{eq:2}, we get,
    \begin{align}
        \pr{EF'} & = \pr{E} - \pr{E} \times \pr{F}
        \\
                 & = \pr{E}\brak{1 - \pr{F}}
        \\
        \label{eq:result}
        \pr{EF'} & = \pr{E} \times \pr{F'}
    \end{align}
    By \eqref{eq:result}, it can be concluded that $E$ and $F'$ are independent events.
\end{frame}

%\section{Code}
%\begin{frame}{Code}
%	Output of the program used to verify whether the solution is correct:
%	\begin{figure}[H]
%		\centering
%		\includegraphics[width=\columnwidth]{../figs/prog.png}
%	\end{figure}
%\end{frame}
\end{document}