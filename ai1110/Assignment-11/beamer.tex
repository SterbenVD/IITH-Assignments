\documentclass{beamer}
\usetheme{CambridgeUS}
\usepackage[utf8]{inputenc}
\usepackage{graphicx}
\usepackage{amsmath}
\usepackage{mathrsfs}
\usepackage{stfloats}
\usepackage{bm}
\usepackage{cite}
\usepackage{cases}
\usepackage{subfig}
\usepackage{amsfonts}
\usepackage{amssymb}
\usepackage{mathtools}
\usepackage{tikz}
\usepackage{circuitikz}
\usepackage{verbatim}
\usepackage{calc}
\usepackage{float}
\usepackage{longtable}
\usepackage{multirow}
\usepackage{multicol}
\usepackage{color}
\usepackage{array}
\usepackage{hhline}
\usepackage{ifthen}
\usepackage{chngcntr}
\usepackage{hyperref}

\newcommand{\BEQA}{\begin{eqnarray}}
\newcommand{\EEQA}{\end{eqnarray}}
\newcommand{\define}{\stackrel{\triangle}{=}}
\def\inputGnumericTable{}

\let\vec\mathbf
\providecommand{\ovl}[1]{\ensuremath{\overline{#1}}}
\providecommand{\pr}[1]{\ensuremath{\Pr\left(#1\right)}}
\providecommand{\sbrak}[1]{\ensuremath{{}\left[#1\right]}}
\providecommand{\lsbrak}[1]{\ensuremath{{}\left[#1\right.}}
\providecommand{\rsbrak}[1]{\ensuremath{{}\left.#1\right]}}
\providecommand{\brak}[1]{\ensuremath{\left(#1\right)}}
\providecommand{\lbrak}[1]{\ensuremath{\left(#1\right.}}
\providecommand{\rbrak}[1]{\ensuremath{\left.#1\right)}}
\providecommand{\cbrak}[1]{\ensuremath{\left\{#1\right\}}}
\providecommand{\lcbrak}[1]{\ensuremath{\left\{#1\right.}}
\providecommand{\rcbrak}[1]{\ensuremath{\left.#1\right\}}}
\providecommand{\abs}[1]{\left\vert#1\right\vert}
\providecommand{\res}[1]{\Res\displaylimits_{#1}}
\newcommand{\myvec}[1]{\ensuremath{\begin{pmatrix}#1\end{pmatrix}}}
\newcommand{\mydet}[1]{\ensuremath{\begin{vmatrix}#1\end{vmatrix}}}

\title{AI1110: Probability and Random Variables}
\subtitle{Assignment 11}
\author{Vishal Vijay Devadiga (CS21BTECH11061)}
\date{\today}
\begin{document}

% Title page frame
\begin{frame}
    \titlepage
\end{frame}

% Remove logo from the next slides
\logo{}

% Outline frame
\begin{frame}{Outline}
    \tableofcontents
\end{frame}

% Beginning of Actual Doc
\section{Question}
\begin{frame}{Question}
    Every Stochastic matrix corresponds to a Markov chain for which it is the one-step transition matrix. \\
    Show that not every Stochastic matrix can correspond to the two-step transition matrix of a Markov chain. \\ 
    In particular, a 2 x 2 stochastic matrix is the two-step transition matrix of a Markov chain if and only if the sum of its diagonal elements is greater than or equal to Unity.
\end{frame}

\section{Solution}
\subsection{Theory}
\begin{frame}{Theory}
    If a stochastic matrix $A =\myvec{a_{ij}}$ where $a_{ij} > 0$, 
    corresponds to the two-step transition matrix of a Markov chain, 
    then there must exist another stochastic matrix $P =\myvec{p_{ij}}$ where $p_{ij} > 0$ 
    such that:
    \begin{align}
        A = P^2
    \end{align}
    Also,
    \begin{align}
        \sum^{ }_{j} p_{ij} = 1
    \end{align}
\end{frame}

\subsection{Proof}
\begin{frame}{Proof}
    In a two state chain, let $P = \myvec{\alpha & 1 - \alpha \\ 1 - \beta & \beta}$
    so that,
    \begin{align}
        A = P^2 =
        \myvec{\alpha^2 + (1 - \alpha)(1 - \beta ) & (\alpha + \beta )(1 - \alpha) \\ (\alpha + \beta )(1 - \beta) & \beta^2 + (1 - \alpha)(1 - \beta)}
    \end{align}
    This gives the sum of this its diagonal entries to be:
    \begin{align}
        a_{11} + a_{22} &= \alpha^2 + 2(1 - \alpha)(1 - \beta )+ \beta^2
        \\
        &= (\alpha + \beta)^2 - 2(\alpha + \beta ) + 2 
        \\
        \label{eq:condition}
        &= 1+(\alpha + \beta - 1)^2 \geq 1 
    \end{align}
    Hence proved.
\end{frame}
\end{document}