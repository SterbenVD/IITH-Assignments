\documentclass{beamer}
\usetheme{CambridgeUS}
\usepackage[utf8]{inputenc}
\usepackage{graphicx}
\usepackage{amsmath}
\usepackage{mathrsfs}
\usepackage{stfloats}
\usepackage{bm}
\usepackage{cite}
\usepackage{cases}
\usepackage{subfig}
\usepackage{amsfonts}
\usepackage{amssymb}
\usepackage{mathtools}
\usepackage{tikz}
\usepackage{circuitikz}
\usepackage{verbatim}
\usepackage{calc}
\usepackage{float}
\usepackage{longtable}
\usepackage{multirow}
\usepackage{multicol}
\usepackage{color}
\usepackage{array}
\usepackage{hhline}
\usepackage{ifthen}
\usepackage{chngcntr}
\usepackage{hyperref}

\newcommand{\BEQA}{\begin{eqnarray}}
\newcommand{\EEQA}{\end{eqnarray}}
\newcommand{\define}{\stackrel{\triangle}{=}}
\def\inputGnumericTable{}

\let\vec\mathbf

\providecommand{\pr}[1]{\ensuremath{\Pr\left(#1\right)}}
\providecommand{\sbrak}[1]{\ensuremath{{}\left[#1\right]}}
\providecommand{\lsbrak}[1]{\ensuremath{{}\left[#1\right.}}
\providecommand{\rsbrak}[1]{\ensuremath{{}\left.#1\right]}}
\providecommand{\brak}[1]{\ensuremath{\left(#1\right)}}
\providecommand{\lbrak}[1]{\ensuremath{\left(#1\right.}}
\providecommand{\rbrak}[1]{\ensuremath{\left.#1\right)}}
\providecommand{\cbrak}[1]{\ensuremath{\left\{#1\right\}}}
\providecommand{\lcbrak}[1]{\ensuremath{\left\{#1\right.}}
\providecommand{\rcbrak}[1]{\ensuremath{\left.#1\right\}}}
\providecommand{\abs}[1]{\left\vert#1\right\vert}
\providecommand{\res}[1]{\Res\displaylimits_{#1}}
\newcommand{\myvec}[1]{\ensuremath{\begin{pmatrix}#1\end{pmatrix}}}
\newcommand{\mydet}[1]{\ensuremath{\begin{vmatrix}#1\end{vmatrix}}}

\title{AI1110: Probability and Random Variables}
\subtitle{Assignment 9}
\author{Vishal Vijay Devadiga (CS21BTECH11061)}
\date{\today}
\begin{document}

% Title page frame
\begin{frame}
    \titlepage
\end{frame}

% Remove logo from the next slides
\logo{}

% Outline frame
\begin{frame}{Outline}
    \tableofcontents
\end{frame}

% Beginning of Actual Doc
\section{Question}
\begin{frame}{Question}
    Given that random variable x is of continuous type. we form the random variable $y = g(x)$.
    \begin{enumerate}
        \item Find $f_y(y)$ if $g(x) = 2F_x(x) + 4$. 
        \item Find $g(x)$ such that y is uniform in the interval $\cbrak{8, 10}$. 
    \end{enumerate}
\end{frame}

\section{Solution}
\subsection{Data given}
\begin{frame}{Data given}
    Given,
    \begin{align}
        \label{eq:data}
        &y = 2F_x(x) + 4 
        \\
        \implies &4 \leq y \leq 6
    \end{align}
    Also,
    \begin{align}
        &g(x) = 2F_x(x) + 4 
        \\ 
        &g'(x) = 2f_x(x) 
    \end{align}
\end{frame}

\subsection{Solution}
\begin{frame}{Solution}

If $4 \leq y \leq 6$ and $y = 2F_x(x) + 4$ has a unique solution $x_1$ then,
\begin{align}
    f_y(y) = \frac{f_x(x_1)}{2f_x(x_1)} = \fbox{0.5}
\end{align}
If $8 < y < 10$, then $y = 2F_x(x) + 8$
\end{frame} 

\end{document}