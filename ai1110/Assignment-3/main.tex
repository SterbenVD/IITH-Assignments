\let\negmedspace\undefined
\let\negthickspace\undefined
%\RequirePackage{amsmath}
\documentclass[journal,12pt,twocolumn]{IEEEtran}
 \usepackage[utf8]{inputenc}
 \usepackage{graphicx}
 \usepackage{amsmath}
 \usepackage{mathrsfs}
\usepackage{txfonts}
\usepackage{stfloats}
\usepackage{bm}
\usepackage{cite}
\usepackage{cases}
\usepackage{subfig}
 \usepackage{amsfonts}
 \usepackage{amssymb}
 \usepackage{enumitem}
\usepackage{mathtools}
\usepackage{tikz}
\usepackage{circuitikz}
\usepackage{verbatim}
\usepackage[breaklinks=false,hidelinks]{hyperref}
\usepackage{listings}
\usepackage{calc}
\usepackage{float}
\usepackage{longtable}
\usepackage{multirow}
\usepackage{multicol}
\usepackage{color}
\usepackage{array}
\usepackage{hhline}
\usepackage{ifthen}
\usepackage{chngcntr}

\newcommand{\BEQA}{\begin{eqnarray}}
\newcommand{\EEQA}{\end{eqnarray}}
\newcommand{\define}{\stackrel{\triangle}{=}}
\bibliographystyle{IEEEtran}
%\bibliographystyle{ieeetr}
\def\inputGnumericTable{}

\let\vec\mathbf

\providecommand{\pr}[1]{\ensuremath{\Pr\left(#1\right)}}
\providecommand{\sbrak}[1]{\ensuremath{{}\left[#1\right]}}
\providecommand{\lsbrak}[1]{\ensuremath{{}\left[#1\right.}}
\providecommand{\rsbrak}[1]{\ensuremath{{}\left.#1\right]}}
\providecommand{\brak}[1]{\ensuremath{\left(#1\right)}}
\providecommand{\lbrak}[1]{\ensuremath{\left(#1\right.}}
\providecommand{\rbrak}[1]{\ensuremath{\left.#1\right)}}
\providecommand{\cbrak}[1]{\ensuremath{\left\{#1\right\}}}
\providecommand{\lcbrak}[1]{\ensuremath{\left\{#1\right.}}
\providecommand{\rcbrak}[1]{\ensuremath{\left.#1\right\}}}
\providecommand{\abs}[1]{\left\vert#1\right\vert}
\providecommand{\res}[1]{\Res\displaylimits_{#1}}
\newcommand{\myvec}[1]{\ensuremath{\begin{pmatrix}#1\end{pmatrix}}}
\newcommand{\mydet}[1]{\ensuremath{\begin{vmatrix}#1\end{vmatrix}}}

\newcommand{\question}{\noindent \textbf{Question: }}
\newcommand{\solution}{\noindent \textbf{Solution: }}


\title{Assignment 3}
\author{Vishal Vijay Devadiga (CS21BTECH11061)}
\date{}
\begin{document}
% make the title area
\maketitle
\question
Refer to Table 14.7, Chapter 14.
\begin{table}[H]
	\input{table/table.tex}
	\caption{Marks of Students}
	\label{tab1}
\end{table}
Find the probability that a Student Obtained:
\begin{enumerate}[label=(\roman{enumi})]
	\item Less than 20\% in the mathematics test.
	\item Marks 60 or Above
\end{enumerate}
\solution
Let's denote the outcome of the experiment by a random variable $X$ such that it maps to following set of integers, $X\in \sbrak{0 , 100}$. \\
\begin{enumerate}[label=(\roman{enumi})]
\item $X < i$ denotes that the Student has less than i marks such that $i \in \sbrak{0 , 100}$
\begin{align}
	\pr{X < 20}    & = \frac{7}{90}
	\\
	& = \fbox{0.078}
\end{align}
\item $X \geq i$ denotes that the Student has greater than or equal to i marks such that $i \in \sbrak{0 , 100}$
\begin{align}
	\pr{X \geq 60} & = \frac{23}{90} 
	\\
	& = \fbox{0.256}
\end{align}
\end{enumerate}
Output of the program used to verify whether the solution is correct:
\begin{figure}[H]
		\centering
		\includegraphics[width = \columnwidth]{./figure/Output.png}
		\caption{Output of the Program}
		\label{fig1}
\end{figure}
\end{document}