\documentclass{beamer}
\usetheme{CambridgeUS}
\usepackage[utf8]{inputenc}
\usepackage{graphicx}
\usepackage{amsmath}
\usepackage{mathrsfs}
\usepackage{stfloats}
\usepackage{bm}
\usepackage{cite}
\usepackage{cases}
\usepackage{subfig}
\usepackage{amsfonts}
\usepackage{amssymb}
\usepackage{mathtools}
\usepackage{tikz}
\usepackage{circuitikz}
\usepackage{verbatim}
\usepackage{calc}
\usepackage{float}
\usepackage{longtable}
\usepackage{multirow}
\usepackage{multicol}
\usepackage{color}
\usepackage{array}
\usepackage{hhline}
\usepackage{ifthen}
\usepackage{chngcntr}
\usepackage{hyperref}

\newcommand{\BEQA}{\begin{eqnarray}}
\newcommand{\EEQA}{\end{eqnarray}}
\newcommand{\define}{\stackrel{\triangle}{=}}
\def\inputGnumericTable{}

\let\vec\mathbf
\providecommand{\ovl}[1]{\ensuremath{\overline{#1}}}
\providecommand{\pr}[1]{\ensuremath{\Pr\left(#1\right)}}
\providecommand{\sbrak}[1]{\ensuremath{{}\left[#1\right]}}
\providecommand{\lsbrak}[1]{\ensuremath{{}\left[#1\right.}}
\providecommand{\rsbrak}[1]{\ensuremath{{}\left.#1\right]}}
\providecommand{\brak}[1]{\ensuremath{\left(#1\right)}}
\providecommand{\lbrak}[1]{\ensuremath{\left(#1\right.}}
\providecommand{\rbrak}[1]{\ensuremath{\left.#1\right)}}
\providecommand{\cbrak}[1]{\ensuremath{\left\{#1\right\}}}
\providecommand{\lcbrak}[1]{\ensuremath{\left\{#1\right.}}
\providecommand{\rcbrak}[1]{\ensuremath{\left.#1\right\}}}
\providecommand{\abs}[1]{\left\vert#1\right\vert}
\providecommand{\res}[1]{\Res\displaylimits_{#1}}
\newcommand{\myvec}[1]{\ensuremath{\begin{pmatrix}#1\end{pmatrix}}}
\newcommand{\mydet}[1]{\ensuremath{\begin{vmatrix}#1\end{vmatrix}}}

\title{AI1110: Probability and Random Variables}
\subtitle{Assignment 10}
\author{Vishal Vijay Devadiga (CS21BTECH11061)}
\date{\today}
\begin{document}

% Title page frame
\begin{frame}
    \titlepage
\end{frame}

% Remove logo from the next slides
\logo{}

% Outline frame
\begin{frame}{Outline}
    \tableofcontents
\end{frame}

% Beginning of Actual Doc
\section{Question}
\begin{frame}{Question}
    If $\ovl{x}$ is the sample mean and $\ovl{v}$ is the sample variance of $x_i$, 
    if the random variables $x_i$ have the same mean $E\cbrak{x_i} = \eta$ and variance $\sigma_i^2 = \sigma^2$
    then prove:
    \begin{align}
        &E\cbrak{\ovl{x}} = \eta
        \\
        &\sigma^2_{\ovl{x}} = \dfrac{\sigma^2}{n}
    \end{align}
\end{frame}

\section{Solution}
\subsection{Data given}
\begin{frame}{Data given}
    Given,
    \begin{align}
        \label{eq:1}
        &\ovl{x} = \dfrac{1}{n}\sum^{n}_{i=1} x_i
        \\
        \label{eq:2}
        &\ovl{v} = \dfrac{1}{n-1}\sum^{n}_{i=1}\brak{x_i - \ovl{x}}^2
    \end{align}
    Also,
    \begin{align}
        \label{eq:3}
        &E\cbrak{x_i} = \eta
        \\
        \label{eq:4}
        &\sigma_i^2 = \sigma^2
    \end{align}
\end{frame}

\subsection{Solution for 1st Question}
\begin{frame}{Solution}
    By linearity of expected value and by \eqref{eq:1},\eqref{eq:3}
    \begin{align}
        E\cbrak{\ovl{x}} &= \dfrac{1}{n}\sum^{n}_{i=1} E\cbrak{x_i}
        \\
        &= \dfrac{1}{n}\sum^{n}_{i=1} \eta
        \\
        &= \eta
    \end{align}
\end{frame}

\subsection{Solution for 2nd Question}
\begin{frame}{Solution}
    By \eqref{eq:2}, and \eqref{eq:4}
    \begin{align}
        \sigma^2_{\ovl{x}} &= \dfrac{1}{n^2}\sum^{n}_{i=1}\sigma^2_{i} 
        \\
        &= \dfrac{1}{n^2}\sum^{n}_{i=1} \sigma^2
        \\
        &= \dfrac{\sigma^2}{n}
    \end{align}
\end{frame}
\end{document}